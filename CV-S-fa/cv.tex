%------------------------------------
% Dario Taraborelli
% Typesetting your academic CV in LaTeX
%
% URL: http://nitens.org/taraborelli/cvtex
% DISCLAIMER: This template is provided for free and without any guarantee 
% that it will correctly compile on your system if you have a non-standard  
% configuration.
% Some rights reserved: http://creativecommons.org/licenses/by-sa/3.0/
%------------------------------------

%!TEX TS-program = xelatex
%!TEX encoding = UTF-8 Unicode

\documentclass[10pt, a4paper]{article}
\usepackage{fontspec} 

% DOCUMENT LAYOUT
\usepackage{geometry} 
\geometry{a4paper, textwidth=5.5in, textheight=8.5in, marginparsep=7pt, marginparwidth=.6in}
\setlength\parindent{0in}

% FONTS
\usepackage[usenames,dvipsnames]{xcolor}
\usepackage{xunicode}
\usepackage{xltxtra}
\defaultfontfeatures{Mapping=tex-text}
%\setromanfont [Ligatures={Common}, Numbers={OldStyle}, Variant=01]{Linux Libertine O}
%\setmonofont[Scale=0.8]{Monaco}
%%% modified by Karol Kozioł for ShareLaTeX use
\setmainfont[
  Ligatures={Common}, Numbers={OldStyle}, Variant=01,
  BoldFont=LinLibertine_RB.otf,
  ItalicFont=LinLibertine_RI.otf,
  BoldItalicFont=LinLibertine_RBI.otf
]{LinLibertine_R.otf}
\setmonofont[Scale=0.8]{DejaVuSansMono.ttf}

% ---- CUSTOM COMMANDS
\chardef\&="E050
\newcommand{\html}[1]{\href{#1}{\scriptsize\textsc{[html]}}}
\newcommand{\pdf}[1]{\href{#1}{\scriptsize\textsc{[pdf]}}}
\newcommand{\doi}[1]{\href{#1}{\scriptsize\textsc{[doi]}}}
% ---- MARGIN YEARS
\usepackage{marginnote}
\newcommand{\amper{}}{\chardef\amper="E0BD }
\newcommand{\years}[1]{\marginnote{\scriptsize #1}}
\renewcommand*{\raggedleftmarginnote}{}
\setlength{\marginparsep}{7pt}
\reversemarginpar

% HEADINGS
\usepackage{sectsty} 
\usepackage[normalem]{ulem} 
\sectionfont{\mdseries\upshape\Large}
\subsectionfont{\mdseries\scshape\normalsize} 
\subsubsectionfont{\mdseries\upshape\large} 

% PDF SETUP
% ---- FILL IN HERE THE DOC TITLE AND AUTHOR
\usepackage[%dvipdfm, 
bookmarks, colorlinks, breaklinks, 
% ---- FILL IN HERE THE TITLE AND AUTHOR
	pdftitle={Albert Einstein - vita},
	pdfauthor={My name},
	pdfproducer={http://nitens.org/taraborelli/cvtex}
]{hyperref}  
\hypersetup{linkcolor=blue,citecolor=blue,filecolor=black,urlcolor=MidnightBlue} 

\usepackage{xepersian}

% DOCUMENT
\begin{document}
{\LARGE علی عسگری}\\[1cm]
تهران، ج. ا. ایران\\
دانشگاه صنعتی شریف\\
دانشکده مهندسی کامپیوتر\\[.2cm]
تلفن:
{98-913-649-6628+}\\
%Fax: \texttt{609-924-8399}\\[.2cm]
رایانامه: \href{mailto:aliasgari@ce.sharif.edu}{aliasgari@ce.sharif.edu}\\
\textsc{تارنما}: \href{http://ce.sharif.edu/~aliasgari/}{http://ce.sharif.edu/\textasciitilde{}aliasgari/}\\ 
\vfill
% Born:  March 12, 1879---Ulm, Germany\\
%Nationality:  German/American

%%\hrule
%\section*{Current position}
%\emph{Emeritus Professor}, Institute for Advanced Study, Princeton

%%\hrule
%\section*{Areas of specialization}
 %Physics • Relativity theory
\section*{علائق تحقیقاتی}
\begin{itemize}
\item پردازش تصویر
\item مهندسی نرم‌افزار (اتوماسیون)
\item کودیزاین نرم‌افزار و سخت‌افزار
\end{itemize}


%%\hrule
\section*{تحصیلات}
\noindent



\years{1394-1398}\textbf{دانشجوی کارشناسی مهندسی کامپیوتر (معدل:‌ 18/77 از 20)}\\
دانشکده‌ی مهندسی کامپیوتر دانشگاه صنعتی شریف\\

\years{1390-1394}\textbf{دیپلم ریاضی - فیزیک}\\
ج. ا. ایران، اصفهان، دبیرستان شهید اژه‌ای (جزء سازمان ملی پرورش استعدادهای درخشان)\\

\section*{تجربه تدریس و کار}
\noindent



\years{1396-1397}\textbf{تحقیق و توسعه پردازش تصویر در  \href{http://me.ut.ac.ir/sne/}{آزمایشگاه نانومهندسی سطح}}\\
دانشکده‌ی مهندسی مکانیک دانشگاه تهران، تحت نظارت فرشید چینی\\

\years{1396 تا الان}\textbf{توسعه دهنده‌ی لینوکس در شرکت سخت‌افزاری-نرم‌افزاری نآد}\\
مرکز خدمات فنآوری دانشگاه صنعتی شریف، تحت نظارت سیاوش بیات سرمدی\\

\years{1395-1396}\textbf{دستیار آموزشی برنامه‌نویسی پیشرفته}\\
پردیس فنآوری اطلاعات دانشگاه صنعتی شریف\\

\years{1396-1397}\textbf{برگزارکننده‌ی فنی (مسئول تیم سایت)}\\
رقابت‌های هوش مصنوعی دانشگاه صنعتی شریف\\

\years{1395-1396}\textbf{برگزارکننده‌ی فنی}\\
رقابت‌های هوش مصنوعی دانشگاه صنعتی شریف\\

\years{1392-1396}\textbf{فریلنسر توسعه‌دهنده نرم‌افزارهای دسکتاپ و وب}

\section*{زبان‌ها}
\noindent
انگلیسی (متوسط\LTRfootnote{Intermediate})، فارسی (زبان مادری)، عربی (مبتدی)\\

\section*{درس‌های مرتبط}
\noindent



\years{پاییز 1396}\textbf{سیگنال‌ها و سیستم‌ها}\\
دانشکده‌ی مهندسی کامپیوتر\\
ج. ا. ایران،‌ تهران،‌ دانشگاه صنعتی شریف\\

\years{پاییز 1396}\textbf{ریاضیات مهندسی}\\
دانشکده‌ی ریاضی،‌ دانشگاه صنعتی شریف\\

\years{پاییز 1396}\textbf{طراحی سیستم‌های دیجیتال}\\
دانشکده‌ی مهندسی کامپیوتر،‌ دانشگاه صنعتی شریف\\

\years{پاییز 1395}\textbf{معماری کامپیوتر (20/00 از 20)}\\
دانشکده‌ی مهندسی کامپیوتر،‌ دانشگاه صنعتی شریف\\

\years{پاییز 1395}\textbf{آمار و احتمال مهندسی (20/00 از 20)}\\
دانشکده‌ی مهندسی کامپیوتر،‌ دانشگاه صنعتی شریف\\


\years{پاییز 1395}\textbf{ساختار زبان و کامپیوتر (19/4 از 20)}\\
دانشکده‌ی مهندسی کامپیوتر،‌ دانشگاه صنعتی شریف\\


\section*{پروژه‌های درسی}
\noindent



\years{پاییز 1395}\href {https://github.com/altostratous/simpu}{طراحی و شبیه‌سازی یک پردازنده‌ی خط لوله‌ی ساده}

\section*{مهارت‌ها}
\noindent
تسلط بر زبان‌های برنامه‌نویسی:‌ \lr{Java, C, C++, Verilog, Python, C\#, php, SQL}\\
کار با بسترها و ابزارها: \lr{OpenCV, Matlab, Django, CakePhp}\\
کار با سیستم‌های عامل:‌ \lr{Linux, Windows}\\
ابزارهای کار با اسناد:‌ \LaTeX

%\vspace{1cm}
\vfill{}
%\hrulefill

\begin{center}
{\scriptsize  آخرین بروزرسانی: \today %\- •\- 
% ---- PLEASE LEAVE THIS BACKLINK FOR ATTRIBUTION AS PER CC-LICENSE
%Typeset in \href{http://nitens.org/taraborelli/cvtex}{
%\fontspec{Times New Roman}
%\XeTeX 
}\\
% ---- FILL IN THE FULL URL TO YOUR CV HERE
%\href{http://nitens.org/taraborelli/cvtex}{http://nitens.org/taraborelli/cvtex}}
\end{center}

\end{document}